% !TEX root = ./tech-specification.tex

%%%%%%%%%%%%%%%%%%%%%%%%%%%%%%%%%%%%%%%%

\section{背景介绍}

%\addcontentsline{toc}{section}{Introduction} % Adds this section to the table of contents

自比特币的诞生起,各类区块链项目取力于无需许可、无需信任的共识机制,创下非凡成功。
比特币之后,公认下最为成功的区块链系统当属以太坊。以太坊将区块链的概念从一个专用的“价值转移系统”扩展为图灵完备的状态机,从而在概念上普适地允许所有种类的运算。
这种普适性的状态机被称为“以太坊虚拟机”(EVM),它本质上将整个以太坊网络作为一个去中心化的计算平台,并根据交易\footnote{这里的“交易”指的是 transaction,在传统计算机系统的语境里也被译作“事务”。本文我们遵循区块链领域的习惯仍将其译为“交易”。}执行情况更新平台内部的“状态”。
尽管以太坊常被誉为无人可关闭的“世界电脑”,但是实际上以太坊底层共识机制过低的吞吐率已经越来越成为整个系统处理能力最大的瓶颈。


比特币共识吞吐率的数学期望是平均每十分钟产生一个大小为 $1$MB 的区块(在有隔离见证的情况下区块容量可以扩大到 $2$MB)。
出于对安全性的考虑,比特币设置了较小的区块尺寸,并维持低吞吐率。
直观而论,在没有恶意攻击的情况下,区块链产生分叉的自然可能性正比于新区块广播时间除以出块时间。因为在新区块广播传遍整个网络前,诚实矿工可能根据最长链规则继续在分叉区块上开采。
%原翻译:比特币的(预期)共识吞吐率是平均每十分钟产生一个大小为 $1$MB 的区块
为实现较短出块时间(约15秒),以太坊采用简化版的 GHOST 规则\cite{GHOST}确定主链,并将区块体积缩小至 $100$KB 左右以缩短广播时间。
包容性区块链协议(Inclusive Block Chain Protocol)\cite{Inclusive}作为一个“区块的有向无环图”协议,使用有向无环图结构(而非链状结构)对网络中的所有区块顺序进行定义。与 GHOST 规则相比,其主要优势在于所有分叉区块都能对共识吞吐率产生贡献。
此外,还存在一系列为提高可扩展性而降低去中心化程度或安全性的提案,例如数据分片、侧链以及第二层延展方案。
而一些极端案例实际执行了中心化的、有准入限制的共识系统。


{\name} 旨在完全保障安全性和去中心化的前提下搭建一个高吞吐率的第一层共识系统。该系统将成为一个普适的、每秒可安全运行数千次交易的计算平台,从而让共识的吞吐率不再成为瓶颈。
我们将{\name} 定位为一个强大的骨干共识网络,一片孕育并助力全新应用、多样扩展的土壤。
在技术层面上,我们借鉴了 \cite{Inclusive} 的设计理念,但采用独创的\tg 结构组织区块,保证了 Conflux 协议的高效实现。