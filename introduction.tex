% !TEX root = ./tech-specification.tex

%%%%%%%%%%%%%%%%%%%%%%%%%%%%%%%%%%%%%%%%

\section{Introduction}

%\addcontentsline{toc}{section}{Introduction} % Adds this section to the table of contents

Since the born of Bitcoin, various blockchain projects have demonstrated extraordinary success with the power of consensus among permissionless and trustless parties.
The most successful blockchain project after Bitcoin is widely considered to be Ethereum, which generalizes the blockchain paradigm from a specialized value-transfer system to a more generalized Turing-complete state machine that allows conceptually all kinds of computation.
This generalized state machine, known as \emph{Ethereum Virtual Machine} (EVM), makes the Ethereum network essentially a decentralized computing platform 
where the state advances on input of transactions.
Sometimes Ethereum is referred to as the ``world computer'' that nobody can shut down, 
except that its processing power is rather poor and severely bounded by the throughput of underlying consensus.

The consensus throughput of Bitcoin is (in expectation) one block per $10$ minutes, with block size $1$MB (or $2$MB with Segregated Witness (segwit)).
Bitcoin is set to small block size and low generation rate mainly for security concerns.
Intuitively, when there is no adversary, the natural probability of forks is proportional to the ratio of  block broadcasting time to block (generation) time,
since under the longest chain rule honest mining power may keep working on a fork during the propagation of a newly mined block.
Ethereum applies a tailored version of GHOST rule \cite{GHOST} and smaller block size to achieve a much shorter block time, i.e. roughly $<100$KB per $15$ seconds.
Inclusive Block Chain Protocol \cite{Inclusive} is a ``block-DAG'' proposal which defines a total order of blocks in a directed acyclic graph (DAG) rather than a chain, with the major advantage over GHOST that all forked blocks contribute to the consensus throughput as well. 
Another line of scaling techniques trades security and decentralization for scalability by using sharding, sidechains, or other second layer extensions.
In extreme cases, centralized and somehow permissioned consensus systems are implemented in practice.


{\name} is a project which aims at building a high throughput first layer consensus system without any compromises in security and decentralization; a generalized computation platform that securely processes at least thousands of transactions per second which makes the  throughput of consensus is no more a bottleneck.
The positioning of {\name} is a strong backbone consensus network on which a numerous number of unprecedented applications and extensions can germinate and thrive.
Technically, we follow a similar idea as \cite{Inclusive} but organize blocks in a \tg, 
which enables a fast implementation of the {\name} protocol.