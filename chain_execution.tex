% !TEX root = ./tech-specification.tex

%%%%%%%%%%%%%%%%%%%%%%%%%%%%%%%%%%%%%%%%

\section{Blockchain Execution}

After determining the total order of blocks, the transactions are executed as if they are packed into sequential blocks on an Ethereum-like chain. 

Blockchain execution is based on a series of ordered blocks $\mathbf{L}$ and a subsequence of pivot blocks $\mathbf{P}$ output by figure~\ref{fig:order}. 
%
The pivot blocks divided into $\mathbf{L}$ into several epochs.  For $k\ge 1$, the epoch $k$ (denoted by $\mathbf{E}_k$) refers the slice in $\mathbf{L}$ started with the next block of $\mathbf{P}[k-1]$ and ended at block $\mathbf{P}[k]$. The epoch 0 refers the genesis block. 


\subsection{Initial state}

The initialization world state $\st^0$ is set as follows. A list $\vec{a}$ with elements $(a,b)$ gives the addresses $a$ and their balance $b$ when the \name blockchain launched.  
\begin{align}
	\forall (a,b)\in \vec{a}, \st^0[a]= \account^0 \quad \mbox{except:} \account_b=b 
\end{align}

In Oceanus, this list contains the following two addresses for faucet. Each address has initialization balance $5\times 10^{33}$ \unit, i.e., 5000 trillion \coinsign.
%
\begin{align*}
	\mathsf{0x1be45681ac6c53d5a40475f7526bac1fe7590fb8} \\
	\mathsf{0x1e768d12395c8abfdedf7b1aeb0dd1d27d5e2a7f}
\end{align*}

The \name internal contracts are also initialized.
%
\begin{align}
	\forall a\in \{a_{\sf stake},a_{\sf sponsor},a_{\sf admin}\}, \st^0[a] &\eqdef \account^0 \quad \mbox{except:} \account_n=1 \\
	\mbox{where:}&\\
	a_{\sf admin} &\eqdef \admincontract \\ 
	a_{\sf sponsor} &\eqdef \sponsorcontract \\
	a_{\sf stake} &\eqdef \stakingcontract
\end{align}

Besides, the global statistic information will be set as follows:

\begin{align}
	\st^0[a_{\sf stake}][k_1]_v & \eqdef \blockinyear\times 2^{80} \\ 
	\st^0[a_{\sf stake}][k_2]_v & \eqdef \blockinyear\times 40000 \\
	\st^0[a_{\sf stake}][k_3]_v & \eqdef 0 \\
	\st^0[a_{\sf stake}][k_4]_v & \eqdef 0 \\
	\st^0[a_{\sf stake}][k_5]_v & \eqdef \sum\nolimits_{(a,b)\in \vec{a}} b \\
	\mbox{where:}&\\
	a_{\sf stake} &\eqdef \stakingcontract \\
	k_1 &\eqdef \sf [accumulate\char`_interest\char`_rate]_{\sf ch} \\ 
	k_2 &\eqdef \sf [interest\char`_rate]_{\sf ch} \\
    k_3 &\eqdef \sf [total\char`_staking\char`_tokens]_{\sf ch} \\
    k_4 &\eqdef \sf [total\char`_storage\char`_tokens]_{\sf ch} \\
    k_5 &\eqdef \sf [total\char`_issued\char`_tokens]_{\sf ch} 
\end{align}


\subsection{Epoch execution}

The blockchain is executed epoch by epoch started with epoch 1. Let $\st_{k-1}$ denote the world state after the execution of epoch $k-1$. The Conflux protocol updates world state from $\st_{k-1}$ to $\st_k$ as follows. Besides updating the world state, the protocol also generates a receipt list $\mathbf{R}_k$ for epoch execution. 

\paragraph{Blocks execution. } First, all the blocks in epoch are executed in sequence by block execution function $\transition_{\sf block}(\st,\block,\mathbf{L}[0..(\tau-1)])=(\st',\mathbf{R}')$, where $\block$ is the block to be executed, $\tau$ is the the index of block $\block$ in $\mathbf{L}$ and $\mathbf{R}'$ is the sequence of transaction receipts. After executing all the blocks, the resultant world-state $\st^*$ becomes the input of the next step and the concatenation of block receipts becomes epoch receipts $\mathbf{R}$. Function $\transition_{\sf block}(\st,\block,\vec{L})$ is defined in section~\ref{sec:block_exec}.

\paragraph{Distribute mining reward. } Since Conflux incentive mechanism puts off the mining reward distribution for $\minerfreeze$ epochs, after execution of epoch $k$, Conflux distribute the mining reward for blocks in $\mathbf{E}_{k-\minerfreeze}$. The computing of mining reward for blocks in epoch $k-\minerfreeze$ requires the following context information.
\begin{itemize}[nosep]
	\item The epoch block set $\mathbf{E}_{k-\minerfreeze}$
	\item The world-state before the execution of all the block $\block$ in $\mathbf{E}_{k-\minerfreeze}$, denoted by $\st(\block)$.
	\item The transaction receipts of all the block $\block$ in $\mathbf{E}_{k-\minerfreeze}$, denoted by $\mathbf{R}'(\block)$.
	\item The tree-graph structure for blocks in $\past(\mathbf{P}[k-\minerfreeze+\anticonecountepoch])$. 
\end{itemize}

Section~\ref{sec:incentive} describes how to compute the block reward $\reward(\block)$ with the context information. The mining reward will be distributed to the block author if the author is . The global parameter \emph{total issued tokens} is updated accordingly. Suppose $\st^*$ is the world state after blocks execution, it will be updated to $\st^{**}$ by
%
\begin{align}
	\st^{**}&\eqdef \st^* \qquad \mbox{except:} \\ 
	\forall a\in \B_{160} \mbox{ with }& \mathsf{Type}_a \in \{[0000]_2,[0001]_2,[1000]_2\} \\ 
	\st^{**}[a]_b&\eqdef \st^*[a]_b +\sum\nolimits_{\block \in \mathbf{E}_{k-\minerfreeze}} \mathbb{I}(\block_{\head_a}=a) \times \reward(\block) \\ 
	\st^{**}[a_{\sf stake}]_{\bf s}[k_3]&\eqdef \st^*[a_{\sf stake}]_{\bf s}[k_3] +\sum\nolimits_{\block \in \mathbf{E}_{k-\minerfreeze}} \left(\mathbb{I}(\mathsf{Type}_{\block_{\head_a}} \in \{[0000]_2,[0001]_2,[1000]_2\} ) \times \reward(\block)-\sum\nolimits_{R\in \mathbf{R}'(\block)}R_f\right) \\ 
	\mbox{where:}&  \\
	a_{\sf stake} &\eqdef \stakingcontract \\ 
	k_3  &\eqdef [{\sf total\char`_issued\char`_tokens}]_{\sf ch}
\end{align}


\subsection{Block execution}\label{sec:block_exec}
The block execution function $\transition_{\sf block}(\st,\block,\vec{L})$ consists of two steps. 

\paragraph{Update accumulate interest}
\begin{align}
	\st^{*}&\eqdef \st \qquad \mbox{except:} \\ 
	\st^{*}[a_{\sf stake}]_{\bf s}[k_1] & \eqdef \left\lfloor\st[a_{\sf stake}]_{\bf s}[k_1] \times \left(1+\frac{4\%}{\blockinyear}\right)\right\rfloor\\
	\mbox{where:}& \\ 
	a_{\sf stake} & \eqdef \stakingcontract \\ 
	k_1 & \eqdef [{\sf accumulate\char`_interest\char`_rate}]_{\sf ch}
\end{align}

\paragraph{Execute transactions in block}

Each block $\block$ contains a series of transactions $\block_{\sf Ts}$. Start with world state $\st^{*}$, \name executes these transactions in sequence by a transform function $\Upsilon(\st,\tx,\vec{L})=(\st',R)$, which updates world state from $\st$ to $\st'$ by processing transaction $\tx$ and outputs a receipt $R$. The input $\vec{L}$ represents the blocks in front of the present block.After executing all the transaction, the resultant state $\st'$ and the concatenation of all the transaction receipts $\mathbf{R}'$ consists of the output of function $\transition_{\sf block}$.
