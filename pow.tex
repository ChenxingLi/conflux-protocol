% !TEX root=./tech-specification.tex

\section{Proof of Work}
\label{sec:pow}



% \subsection{Proof of Work Function}

\oldversion{In the Pontus version, {\name} tentatively applies a twisted tripple Keccack as the proof of work function $\pow$.
It will be updated in the next stage of {\name} mainnet launch.

More specifically, the function $\pow$ is defined on block headers as follows:
\begin{align}
	\pow\left( \head \right) \eqdef 
	\kec\big(\kec\left(\kec\left(\rlp( \head_{-n} )\right)\circ \head_n\right) 
	\oplus \kec\left(\rlp( \head_{-n})\right)
	% \rlp(\head_{-n})[0\dots 255]
	\big)
\end{align}
}

{\name} applies the \emph{Multi-point Ethash} function $\mpethash$
as the Proof-of-Work function $\pow$.
$\mpethash$ is a twisted version of \textsf{Ethash} function with additional evaulation of a polynomial on multiple points.
The detailed specification of this function is in Appendix~\ref{app:mp_eval_hash}.

The $\mpethash$ function is defined as:
\begin{equation}\label{eq:mpethash}
	\mpethash(\head)
	\eqdef \mpethash\left(\kec\left(\rlp( \head_{-n} )\right), \head_{n},\dataset\right) 
	\eqdef 
	\set{\compressedmix, \kec\left(\seedhash \circ \compressedmix \circ \mpmix \right)}
\end{equation}
where $\dataset$ denotes the dataset derived from $\head$ as in Appendix~\ref{app:dataset} and $\head_{-n}$ denotes the header excluding the {\bf mixHash} and {\bf nonce} fields,
i.e. $\head \equiv \head_{-n} \circ \head_x \circ \head_n$ since {\bf mixHash} and {\bf nonce} are indeed the last two fields in the structure of block header,
and the fields of $\head_{-n}$ are \rlp-serialized according to their order in Section~\ref{sec:block}. 

The output of $\mpethash$ consists of an array with the compressed mix $\compressedmix$ as its first item, 
and the Keccak-256 hash of the concatenation of the seed hash $\seedhash\in \B_{512}$, the compressed mix $\compressedmix\in \B_{256}$,
and the compressed multi-point mix $\mpmix\in\B_{32}$ as the second item.
See Appendix~\ref{appsec:pow} for more details.

The first output item of $\mpethash$ is used as the Proof-of-Work function $\pow$, 
and the second output item is realized as the mix-hash $\head_x$:
\begin{align}\label{eq:pow}
	% \mpethash(\head)
	% &\eqdef \mpethash\left(\kec\left(\rlp( \head_{-n} )\right), \head_{n},\dataset\right) 
	% \eqdef 
	% \set{\compressedmix, \kec\left(\seedhash \circ \compressedmix \circ \mpmix \right)}\\
	\pow(\head) 
	&\eqdef \mpethash\left(\kec\left(\rlp( \head_{-n} )\right), \head_{n},\dataset\right)[1] 
	= \kec\left(\seedhash \circ \compressedmix \circ \mpmix\right)\\
	\head_x 
	&\eqdef \mpethash\left(\kec\left(\rlp( \head_{-n} )\right), \head_{n},\dataset\right)[0] 
	= \compressedmix \label{eq:mixhash}
\end{align}




% \subsection{The PoW Puzzle [new name?]}

% \lipsum[10-12] % Dummy text

\subsection{Proof-of-Work Quality}
\label{subsec:quality}

The \emph{proof-of-work quality} (a.k.a. \emph{PoW quality} or simply \emph{quality}) of a block refers to the expected amount of work spent in finding such a block.
Given a block $\block$ with header $\head(\block)$ and 
the $256$-bit scalar $\offset(\head) \eqdef \left[\head_n(\block)[1\dots127]\right]_2\times 2^{128} \in \N_{256}$ which denotes the offset of proof-of-work validation,
the quality of $\block$ essentially represents the expected number of random trials to find a block $\block'$ with header $\head'$ satisfying that $\pow(\head')$ is in between of $\offset(\head)$ and $\pow(\head)$.

More specifically, the block $\block$ with header $\head(\block)$ has quality
\begin{align}
	\quality(\block)\eqdef 
	\quality(\head) \eqdef
	\begin{cases}
		\left\lfloor2^{256}/\left(\pow(\head) - \offset(\head) +1 \right) \right\rfloor 
		& \mbox{if $\pow(\head) > \offset(\head)$}\\
		\left\lfloor2^{256}/\left(2^{256}+\pow(\head) - \offset(\head) +1 \right) \right\rfloor 
		& \mbox{if $\pow(\head) < \offset(\head)$}\\
		2^{256}-1 &  \mbox{if $\pow(\head) = \offset(\head)$}
	\end{cases}
\end{align}


\subsection{Difficulty Adjustment}
\label{sec:difficulty}


The difficulty is adjusted according to the block generation rate in the past.
More specifically,
we estimate the current computing power of all miners from the number of blocks in the last $\difficultyadjustperiod$ epochs and the average timestamps of blocks in the beginning and ending epochs,
and then set the target difficulty for the next $\difficultyadjustperiod$ epochs such that the expected block generation rate should be roughly one block per $\blocktime$ seconds.

Formally,
for $0\le j\le \difficultyadjustperiod$,
the target difficulty of a block at height $j$ is set to $\mathbf{d}_0\eqdef {\startdifficulty=\startdifficultyline}$;
for any positive integer $i\ge 1$,
the target difficulty of blocks at height $j\in [\difficultyadjustperiod i+1,\difficultyadjustperiod i+\difficultyadjustperiod]$ is set to $\mathbf{d}_i \in\N_{256}$ such that

\newversion{
	\begin{align}
		\mathbf{d}_i \eqdef 
		\begin{cases}
			\lfloor \mathbf{d}_{i-1} \times \diffup \rfloor & \text{if $\mathbf{d}'_i> \mathbf{d}_{i-1} \times \diffup$}\\
			\lceil \mathbf{d}_{i-1} \times \diffdown \rceil & \text{if $\mindifficulty\le \mathbf{d}'_i< \mathbf{d}_{i-1} \times \diffdown$}\\
			\mindifficulty & \text{if $\mathbf{d}'_i < \mindifficulty$}\\
			\mathbf{d}'_{i} & \text{otherwise ($\mathbf{d}_{i-1} \times \diffdown \le \mathbf{d}'_i \le \mathbf{d}_{i-1} \times \diffup$)}
		\end{cases}
	\end{align}
	where $\mathbf{d}'_i\in\N_{256}$ is the estimation of ideal target difficulty defined as follows
	\begin{align}
		% \mathbf{d}'_i\eqdef \mathbf{d}_{i-1} \times \blocktime \times {\sum_{j=1}^{\difficultyadjustperiod}|\epoch_{\difficultyadjustperiod i+j}|}/
		% {\left(\frac{\sum_{\block'\in\epoch_{\difficultyadjustperiod i}} \block'_{\head_s}}{|\epoch_{\difficultyadjustperiod i}|} - \frac{\sum_{\block'\in\epoch_{\difficultyadjustperiod(i-1)}} \block'_{\head_s}}{|\epoch_{\difficultyadjustperiod(i-1)}|} \right)}
		\mathbf{d}'_i\eqdef \mathbf{d}_{i-1} \times \blocktimeunix \times 
		\left({\sum_{j=\difficultyadjustperiod (i-1)+1}^{\difficultyadjustperiod i}|\epoch_{j}|} -1\right)/
		{\left( \head\left(\block^{(\difficultyadjustperiod i)}\right)_{s} - \min_{\difficultyadjustperiod (i-1)\le j< \difficultyadjustperiod i \;\land\; \head\left(\block^{\left(j\right)}\right)_{s}\ne 0}\set{ \head\left(\block^{\left(j\right)}\right)_{s} } \right)}
	\end{align}
	where for every $k\in\N$, $\epoch_k$ denotes the set of \emph{fully valid} blocks in the $k$-th epoch and $\block^{(k)}$ denotes the pivot block in $\epoch_k$.
	In the above formula the total number of blocks in last $\difficultyadjustperiod$ epochs is decreased by $1$, which leads to an unbiased estimation of block generation rate.
}

Note that a single block $\block$ may not have a global view.
Indeed, the best it could do is to compute the target difficulty $\mathbf{d}_i$ from its local view of blocks in $\past(\block)$.
In particular, a block $\block$ at height $h\eqdef \head\left(\block\right)_h$ should have target difficulty
\begin{align}
	\head\left(\block\right)_d\eqdef \begin{cases}
		\mathbf{d}_0 & h= 0\\
		\mathbf{d}_{\left\lfloor \frac{ h-1}{\difficultyadjustperiod} \right\rfloor} & \mbox{$h>0$, $\mathbf{d}_{\left\lfloor \frac{ h-1}{\difficultyadjustperiod} \right\rfloor}$ is  calculated with respect to $\past(\block)$}
	\end{cases} 
\end{align}

\paragraph{Epoch difficulty.}
As soon as all nodes agree on the pivot block $\block^{(k)}$ at the $k$-th epoch $\epoch_k$,
we can uniquely define the target difficulty of $\epoch_k$ as the target difficulty of $\block^{(k)}$.
Formally, 
\begin{align}
	\mathbf{d}_{\epoch_k}\eqdef \head\left( \block^{(k)} \right)_d
\end{align}
where $ \head\left( \block^{(k)} \right)_d$ is the {\bf difficulty} field in $\block^{(k)}$'s header and it equals to $\mathbf{d}_{\left\lfloor \frac{k-1}{\difficultyadjustperiod} \right\rfloor}$ derived from the past view of $\block^{(k)}$.


% \paragraph{Threshold condition.}
% 	We say that a block $\block$ fulfills the threshold condition of difficulty adjustment if $\mathbf{d}\eqdef\block_{\head_d}$ and $\mathbf{d}'\eqdef\parent{\block}_{\head_d}$ satisfy the following:
% 	\begin{align}
% 		\begin{cases}
% 			\mathbf{d}'\times\diffdown \le \mathbf{d} \le \mathbf{d}'\times \diffup & \text{if $h\eqdef \head\left(\block\right)_h$ is a height for difficulty adjustment, i.e.~$h> \difficultyadjustperiod$ and $\difficultyadjustperiod \;|\; (h-1)$}\\
% 			\mathbf{d} = \mathbf{d}' & \text{otherwise}
% 		\end{cases}
% 	\end{align}


% \subsection{Definition of Work}

% \lipsum[16-17] % Dummy text






